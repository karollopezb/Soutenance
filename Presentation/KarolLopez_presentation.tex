% !TEX encoding = UTF-8 Unicode
%%%%%%%%%%%%%%%%%%%%%%%%%%%%%%%%%%%%%%%%%
% Beamer Presentation
% LaTeX Template
% Version 1.0 (10/11/12)
%
% This template has been downloaded from:
% http://www.LaTeXTemplates.com
%
% License:
% CC BY-NC-SA 3.0 (http://creativecommons.org/licenses/by-nc-sa/3.0/)
%
%%%%%%%%%%%%%%%%%%%%%%%%%%%%%%%%%%%%%%%%%

%----------------------------------------------------------------------------------------
%	PACKAGES AND THEMES
%----------------------------------------------------------------------------------------
\documentclass[french]{beamer}
\usepackage{beamerthemesplit}
\usepackage[utf8]{inputenc}
\usepackage[T1]{fontenc}
\usepackage{anyfontsize}
\usepackage{amsmath,amssymb,amsfonts,amsbsy}
\usepackage{eqnarray,amsmath}
\usepackage{booktabs}
\usepackage{subfig}
\usepackage{multirow}
\usepackage{booktabs}
\usepackage{dcolumn}
% algorithms
\usepackage{algorithmic}
\usepackage{algorithm}
% mathematical
% 
\usepackage{bibentry}
\nobibliography*
% 

\usepackage{amssymb}
\usepackage{pifont}
\usepackage{graphicx}
\usepackage{epstopdf}
\usepackage{animate}
\usepackage{hyperref}


\graphicspath{{fig}}

\epstopdfDeclareGraphicsRule{.gif}{png}{.png}{convert gif:#1 png:\OutputFile}
\AppendGraphicsExtensions{.gif}% http://ctan.org/pkg/pifont
\definecolor{MediumSeaGreen}{rgb}{0.2344,0.6992,0.4414}
\newcommand{\cmark}{\textcolor{MediumSeaGreen}{\ding{52}}}%
\newcommand{\xmark}{\textcolor{red}{\ding{54}}}%
\renewcommand{\multirowsetup}{\centering}		% to use centering in multirow command	%DB
\newcommand{\B}[1]{\textbf{#1}}	% to obtain bold fonts %DB
\newcommand{\DB}[1]{\textcolor{red}{#1}}	% to makes a highlighted text in hcl color
\newcommand{\T}[1]{\textcolor{black}{#1}}	
\graphicspath{{fig/}}
\newcommand{\TOM}[1]{\textcolor{red}{[\textsc{TOM}: \emph{#1}]}}
\newcolumntype{C}[1]{>{\centering\arraybackslash}p{#1}} % to force centered columns with a specific width
%\newcolumntype{L}[1]{>{\raggedright\arraybackslash}p{#1}}
%\newcolumntype{R}[1]{>{\raggedleft\arraybackslash}p{#1}}
\renewcommand{\multirowsetup}{\centering}	% to use centering in multirow command	%DB
\def\s{\hphantom{0}}	% {\s} space corresponding to a number  (for tables centering) %DB
\def\x{\hphantom{$-$}}	% {\x} space corresponding to a negative (for tables centering) 
\newcommand{\tabitem}{~~\llap{\textbullet}~~}



\DeclareGraphicsExtensions{.pdf,.png,.eps,.jpg}
\tolerance=1000
\hyphenpenalty=1000
\overfullrule=1mm
\providecommand{\ve}[1]{{\pmb{#1}}} %
\providecommand{\mat}[1]{{\pmb{#1}}} %
\DeclareMathOperator{\Real}{\mathbb{R}}
\DeclareMathOperator{\Natural}{\mathbb{N}}
\mode<presentation>{

% The Beamer class comes with a number of default slide themes
% which change the colors and layouts of slides. Below this is a list
% of all the themes, uncomment each in turn to see what they look like.

% \usetheme{default}
% \usetheme{AnnArbor}
% \usetheme{Antibes}
%\usetheme{Bergen}
%\usetheme{Berkeley}
% % % % % \usetheme{Berlin}
%\usetheme{Boadilla}
%\usetheme{CambridgeUS}
%\usetheme{Copenhagen}
%\usetheme{Darmstadt}
%\usetheme{Dresden}
%\usetheme{Frankfurt}
%\usetheme{Goettingen}
%\usetheme{Hannover}
%\usetheme{Ilmenau}
%\usetheme{JuanLesPins}
%\usetheme{Luebeck}
\usetheme{Madrid}
%\usetheme{Malmoe}
%\usetheme{Marburg}
%\usetheme{Montpellier}
%\usetheme{PaloAlto}
%\usetheme{Pittsburgh}
%\usetheme{Rochester}
%\usetheme{Singapore}
% \usetheme{Szeged}
%\usetheme{Warsaw}

% As well as themes, the Beamer class has a number of color themes
% for any slide theme. Uncomment each of these in turn to see how it
% changes the colors of your current slide theme.

%\usecolortheme{albatross}
%\usecolortheme{beaver}
%\usecolortheme{beetle}
%\usecolortheme{crane}
%\usecolortheme{dolphin}
%\usecolortheme{dove}
%\usecolortheme{fly}
%\usecolortheme{lily}
%\usecolortheme{orchid}
%\usecolortheme{rose}
%\usecolortheme{seagull}
\usecolortheme{seahorse}
%\usecolortheme{whale}
%\usecolortheme{wolverine}

%\setbeamertemplate{footline} % To remove the footer line in all slides uncomment this line
%\setbeamertemplate{footline}[page number] % To replace the footer line in all slides with a simple slide count uncomment this line

%\setbeamertemplate{navigation symbols}{} % To remove the navigation symbols from the bottom of all slides uncomment this line
}

\usepackage{graphicx} % Allows including images
\usepackage{booktabs} % Allows the use of \toprule, \midrule and \bottomrule in tables

% \usepackage[font=Times,timeinterval=30,timeduration=15,timewarningfirst=80,timewarningsecond=90,colorwarningfirst=blue,fillcolorwarningfirst=white!60!yellow, fillcolorwarningsecond=white!10!yellow, resetatpages=2]{tdclock}


% \addtobeamertemplate{navigation symbols}{}{%
%     \usebeamerfont{footline}%
%     \usebeamercolor[fg]{footline}%
%     \hspace{1em}%
%     \insertframenumber/\inserttotalframenumber
% }

%----------------------------------------------------------------------------------------
%	TITLE PAGE
%----------------------------------------------------------------------------------------

\title[Recharge intelligente de VE]{Une approche d'apprentissage automatique pour la recharge intelligente des véhicules électriques} 
% \title[Recharge intelligente de VE \hspace{5mm} \cronominutes:\cronoseconds]{Une approche d'apprentissage automatique pour la recharge intelligente des véhicules électriques} 

\author[Karol Lina L\'opez]{\texorpdfstring{Karol Lina L\'opez\newline\href{mailto:karol-lina.lopez.1@ulaval.ca}{karol-lina.lopez.1@ulaval.ca}}{Karol Lina L\'opez}}
  \centering
 \titlegraphic{
 \begin{minipage}{0.33\textwidth}
 \includegraphics[width=2.5cm]{REPARTI.jpg}
 \end{minipage}\hfill%
 \begin{minipage}{0.33\textwidth}
 \includegraphics[width=2.5cm]{ul_logo.pdf}
 \end{minipage}\hfill%
 \begin{minipage}{0.33\textwidth}
 \includegraphics[width=2.5cm]{LVSN2.png}
 \end{minipage}
}
\date{22 jan 2019} % Date, can be changed to a custom date


\begin{document}


{\setbeamercolor{block body}{bg=white}
\newcommand{\abs}[1]{\left\vert#1\right\vert}
\newcommand{\argmin}{\operatornamewithlimits{argmin}}
\newcommand{\argmax}{\operatornamewithlimits{argmax}}
\newcommand{\mean}{\operatornamewithlimits{mean}}

\begin{frame}
% \initclock
\titlepage % Print the title page as the first slide
\end{frame}
% 
% \begin{frame}
% \frametitle{Plan de la présentation} % Table of contents slide, comment this block out to remove it
% \fontsize{10}{1}{
% \tableofcontents[hideallsubsections]}% Throughout your presentation, if you choose to use \section{} and \subsection{} commands, these will automatically be printed on this slide as an overview of your presentation
% \end{frame}

% 
% 
% \begin{frame}
% \frametitle{}
% 
% \end{frame}

%----------------------------------------------------------------------------------------
%	PRESENTATION SLIDES
%----------------------------------------------------------------------------------------
%------------------------------------------------
%\section{Pourquoi des ensembles ?}

\section{Introduction}
\subsection{Contexte}

% \begin{frame}
% \frametitle{Prévision de l'achat de VEs}
% \begin{center}
% %\includegraphics<1>[width=0.7\linewidth]{Electric-Cars_sales.png}
% \includegraphics[width=0.7\linewidth]{BNEFforetast.jpg}
% %\includegraphics<2>[width=0.7\linewidth]{PEV-Sales-Canada.jpg}
% %\includegraphics<2>[width=0.7\linewidth]{previsiondemande.jpg}
% \end{center}
% \end{frame}
% 
% \begin{frame}
% \frametitle{Progression des ventes de Véhicules Électriques (VE)}
% \begin{center}
% %\includegraphics<1>[width=0.7\linewidth]{Electric-Cars_sales.png}
% % \includegraphics[width=0.8\linewidth]{salesEV.pdf}\\
% \includegraphics[width=0.8\linewidth]{fig/salesEV.pdf}\\
% \tiny{Source: Cobb, Jeff. 2017. "Top 10 Plug-in Vehicle Adopting Countries of 2016". HybridCars.com. Retrieved 2017-01-23.}
% %\includegraphics<2>[width=0.7\linewidth]{PEV-Sales-Canada.jpg}
% %\includegraphics<2>[width=0.7\linewidth]{previsiondemande.jpg}
% \end{center}
% \end{frame}


\begin{frame}
\frametitle{Véhicules électriques : portrait de la situation}
\vspace{-1.5em}
\begin{center}
\begin{figure}
\includegraphics<1>[width=0.9\linewidth]{evolution_ventes_ve.jpg}
\caption*{\only<1>{\tiny{Source: Bloomberg, 2017}}}

\includegraphics<2>[width=0.55\linewidth]{engagements.png}
\caption*{\only<2>{\tiny{Source: Radio-Canada, 2018}}}
\end{figure}
% \tiny{Source: Inauguration of the European Interoperability Centre for Electric Vehicles and Smart Grids}
% \tiny{Source: Report on the Economic and Environmental Impacts of Large-Scale Introduction on EV/PHEV. Shakoor and Aunedi, 2011}
\end{center}
\end{frame}

\begin{frame}
\frametitle{Intégrer les véhicules électriques dans le réseau électrique}
\begin{center}
\begin{figure}
\includegraphics<1>[width=0.9\linewidth]{smartgrid.png}
\caption*{\only<1>{\tiny{Source: Inauguration of the European Interoperability Centre for Electric Vehicles and Smart Grids}}}

\includegraphics<2>[width=0.6\linewidth]{picos.pdf}
\caption*{\only<2>{\tiny{Source: Report on the Economic and Environmental Impacts of Large-Scale Introduction on EV/PHEV. Shakoor $\&$ Aunedi, 2011}}}
\end{figure}
% \tiny{Source: Inauguration of the European Interoperability Centre for Electric Vehicles and Smart Grids}
% \tiny{Source: Report on the Economic and Environmental Impacts of Large-Scale Introduction on EV/PHEV. Shakoor and Aunedi, 2011}
\end{center}
\end{frame}

% 
% 
% 
% 
\begin{frame}
% %  http://www.intechopen.com/books/electric-vehicles-the-benefits-and-barriers/integration-of-electric-vehicles-in-the-electric-utility-systems
\frametitle{Stratégies pour la gestion de la demande}

\begin{tabular}{l|cc}
 			& \textbf{Orienté grille} & \textbf{Orienté utilisateur} \\ \hline
\emph{Objectifs} 	& 
\begin{minipage}{0.38\linewidth}\vspace{1em}
	\begin{itemize}
		\item Réduire la demande pic en puissance
		\item Réduire les coûts de production
	\end{itemize}\vspace{1em}
\end{minipage} 
& \begin{minipage}{0.38\linewidth}\vspace{1em}
	\begin{itemize}
		\item Assurer la disponibilité du véhicule
		\item Réduire le coût d'opération
	\end{itemize}\vspace{1em}
\end{minipage}  \\ \hline
\emph{Requis} & 
\begin{minipage}{0.38\linewidth}\vspace{1em}
	\begin{itemize}
		\item Données de consommation globales
		\item Infrastructure de communication complexe
	\end{itemize}
\end{minipage} 
& \begin{minipage}{0.38\linewidth}\vspace{1em}
	\begin{itemize}
		\item Données spécifiques à l'utilisateur
		\item Infrastructure de communication minimale
	\end{itemize}
\end{minipage}  \\
\end{tabular}
%
%\begin{columns}
%\begin{column}{0.5\linewidth}
%\begin{center}
%\textbf{Orienté grille}
%\end{center}
%\end{column}
%\begin{column}{0.5\linewidth}
%\begin{center}
%\textbf{Orienté utilisateur}
%\end{center}
%\end{column}
%\end{columns}

% \begin{center}
%\begin{figure}
%\includegraphics[width=\linewidth]{figConcMapFr.pdf}
%\end{figure}
% % \includegraphics<1>[width=0.6\linewidth]{previsiondemande.jpg}
% % \includegraphics[width=0.8\linewidth]{picos.pdf}\\
% \includegraphics[width=0.8\linewidth]{fig/picos.pdf}\\
% \tiny{Source: Report on the Economic and Environmental Impacts of Large-Scale Introduction on EV/PHEV. Shakoor and Aunedi, 2011}
% % \caption*{\only<1>{}\only<2>{\tiny Source: Report on the Economic and Environmental Impacts of Large-Scale Introduction on EV/PHEV. Shakoor $\&$ Aunedi, 2011}}
% \end{figure} 
% %{\tiny Source: Inauguration of the European Interoperability Centre for Electric Vehicles and Smart Grids}
% \end{center}

\end{frame}
% 

\begin{frame}
\frametitle{Stratégies pour la gestion de la demande}
\begin{columns}[T]
\begin{column}[T]{0.5\linewidth}
\begin{overlayarea}{0.95\linewidth}{2.8in}
\only<1->{
\begin{block}{Contrôle direct} 
\vspace{1mm}
\includegraphics[width=\linewidth]{elNet.jpg}
\end{block}
}
\end{overlayarea}
\end{column}

\begin{column}[T]{0.5\linewidth}  %%<--- here
\begin{overlayarea}{0.95\linewidth}{2.8in}
\only<2->{
\begin{block}{Tarification dynamique} 
\includegraphics[width=\linewidth]{mb.jpg}
\end{block}
}
\end{overlayarea}
\end{column}
\end{columns}
\end{frame}


\begin{frame}
\begin{center}
\frametitle{Tarification dynamique}
\includegraphics<1>[width=\linewidth]{hydroquebec.jpg}
\end{center}
\end{frame}

\begin{frame}
\begin{center}
\huge \textbf{Optimiser la recharge de véhicules électriques avec une perspective utilisateur}
\end{center}
\end{frame}


\begin{frame}{Données disponibles}
\begin{center}
\includegraphics[width=0.25\linewidth]{dollarelectricity}\hspace{0.2\linewidth}
\includegraphics[width=0.25\linewidth]{cadran}

\includegraphics[width=0.25\linewidth]{event-date-and-time-symbol}\hspace{0.2\linewidth}
\includegraphics[width=0.25\linewidth]{partly-cloudy-day}
\end{center}
\end{frame}


\begin{frame}{Approches d'apprentissage disponibles}
\includegraphics[width=0.7\linewidth]{neuralnet_stylised-1}
\end{frame}



\begin{frame}{Méthodologie générale}
\includegraphics[width=\linewidth]{schema-global}
\end{frame}

\begin{frame}{Méthodologie générale}
\begin{enumerate}
	\item Obtenir et analyser les données pertinentes
	\item Déterminer les décisions optimales correspondantes
	\item Proposer une méthode d'apprentissage capable d'approcher les décisions optimales
\end{enumerate}
\end{frame}


\section{Analyse des données}

\begin{frame}
\begin{center}
\huge \textbf{Obtenir et analyser les données pertinentes}
\end{center}
\end{frame}

\subsection{Analyse univariée des données}
\begin{frame}
\frametitle{Analyses univariées des données} 
\begin{overlayarea}{\linewidth}{2.8in}
\begin{center} 
\includegraphics<1 | handout:0>[width=0.78\linewidth]{figUnivAnlFr1.pdf} 
\includegraphics<2 | handout:0>[width=0.78\linewidth]{figUnivAnlFr2.pdf}
\includegraphics<3 | handout:0>[width=0.78\linewidth]{figUnivAnlFr3.pdf}
\end{center}
\end{overlayarea}
\end{frame}

\subsection{Analyse multivariée des données}
\begin{frame}
\frametitle{Analyses multivariées des données - Corrélation et ACP} 

\begin{table}[t!]
\begin{center}
\centering
\scalebox{0.65}{
\begin{tabular}{lccccc}
\toprule
{} & \textbf{Température} & \textbf{Énergie consommée} & \textbf{Prix de l'essence} & \textbf{HOD} & \textbf{HOEP} \\
\midrule
\textbf{Température}   	& \x1.00&0.02	&\B{0.81}	&$-$0.02	&0.15	\\
\textbf{Énergie consommée}    & 	&1.00	&0.02	&\x0.06		&0.06	\\
\textbf{Prix de l'essence} & 	&	&1.00	&$-$0.00	&0.17	\\
\textbf{HOD}            & 	& 	&	&\x1.00		&\B{0.71}	\\
\textbf{HOEP}           & 	&	&	&		&1.00	\\
\bottomrule
\end{tabular}
}
\end{center}
\end{table}

\pause

\begin{table}
\begin{center}
\centering
\scalebox{0.65}{
\begin{tabular}{lcccccc}
\toprule
{} & \textbf{1$^{\mbox{st}}$} & \textbf{2$^{\mbox{nd}}$} & 
\textbf{3$^{\mbox{rd}}$} & \textbf{4$^{\mbox{th}}$} & \textbf{5$^{\mbox{th}}$} \\
\midrule
\textbf{Température}   &0.57&$-$0.41&$-$0.01&$-$0.06&\x0.71\\
\textbf{Énergie consommée}    &0.08&\x0.08&\x0.99&\x0.01&\x0.00\\
\textbf{Prix de l'essence} &0.58&$-$0.40&$-$0.02&$-$0.10&$-$0.71\\
\textbf{HOD}                &0.35&\x0.63&$-$0.07&$-$0.69&\x0.03\\
\textbf{HOEP}              &0.47&\x0.52&$-$0.09&\x0.71&$-$0.01\\
\bottomrule
\end{tabular}
}
\end{center}
\end{table}
\end{frame}


\begin{frame}
\frametitle{Définition du Système d'Information} 


\begin{itemize}
\item HOEP ($\ve{x}^1$), HOD ($\ve{x}^2$), et température extérieur ($\ve{x}^3$) with 101 décalages.
\item énergie consommée ($\ve{x}^4$) avec 199 décalages.
\item variables scalaires : 
\begin{itemize}
\item $w_1:$ jour de la semaine,
\item $w_2:$ heure;
\item $w_3: C_\mathit{el}(t-1)-C_\mathit{el}(t)$ différence du prix de l'électricité,
\item $w_4: C_\mathit{el}(t)$ prix de l'électricité au temps $t$ [\$/kWh],
\item $w_5: C_\mathit{fuel}(t)$ prix de l’essence au temps $t$[\$/l],
\item $w_6: dis(t)$ distance parcourue au temps $t$ [km].
\end{itemize}
\end{itemize}

\pause
... et l'état de charge du véhicule ainsi que les décisions optimales obtenues avec la programmation dynamique.

\end{frame}

\begin{frame}
\frametitle{ACP du Système d'Information} 
\begin{figure}
\begin{center}
\includegraphics[width=0.99\textwidth]{figPCAvarExplainedFr.png}
\end{center}
\end{figure}
\end{frame}


\section{Stratification des données}
\subsection{Méthode pour la stratification de données}
\begin{frame}
\frametitle{Stratification des données} 
\begin{figure}
\begin{center}
\includegraphics<1>[width=0.9\linewidth]{figClusteringTrainingFr_222_summer1.pdf}
\includegraphics<2>[width=0.9\linewidth]{figStratEstimFr_222_summer1.pdf}
\end{center}
\end{figure}
\end{frame}
% 
\begin{frame}
\frametitle{Stratification dans un problème de régression avec réseau de neurones} 
\begin{figure}
\begin{center}
\includegraphics[width=0.7\textwidth]{figDiff_ANN_Inp_state_year_2003.pdf}
\end{center}
\end{figure}
\end{frame}

\begin{frame}
\frametitle{Stratification dans un problème de régression avec machines à vecteurs de support} 
\begin{figure}
\begin{center}
\includegraphics[width=0.7\textwidth]{figDiff_SVR_Inp_state_year_2003.pdf}
\end{center}
\end{figure}
\end{frame}




\begin{frame}
\begin{center}
\huge \textbf{Déterminer les périodes de recharge optimales}
\end{center}
\end{frame}


\subsection{Définition de la problématique}


\begin{frame}
\begin{center}
\frametitle{Programmation des périodes de recharge}
\includegraphics<1>[width=0.9\linewidth]{obj01.pdf}
\includegraphics<2>[width=0.9\linewidth]{obj02.pdf}
\includegraphics<3>[width=0.9\linewidth]{obj03.pdf}
\end{center}
\end{frame}


\begin{frame}
\frametitle{Si on suppose qu'on connaît le futur?}
\begin{center}
\begin{figure}
% \includegraphics[width=\linewidth]{minFct.pdf}
\includegraphics[width=\linewidth]{PriceDist_201Fr.pdf}
\end{figure} 
 \end{center}
\end{frame}

\begin{frame}{Formulation du problème}

\begin{itemize}
	\item Discrétisation de la journée et des états de charge
	\item À chaque temps $t$, le véhicule peut être branché ($z(t)=1$) ou non branché ($z(t)=0$)
	\item Si branché, on peut charger ou non ($a(t) = \{0, 1\}$)
\begin{equation}\label{eq:costp1}
S_p(t) = a(t) \cdot C_{el}(t) \cdot \frac{E_{ch}(SoC(t))}{\eta}
\end{equation}
	\item Sinon, on ne peut que consommer de l'énergie
\begin{equation}\label{eq:costp2}
S_u(t) = C_{fuel}(t) \cdot \max(F_{c}(SoC(t)),0)
\end{equation}
	\item Objectif : minimiser le coût d'utilisation
\begin{equation}
\min_{\{a(t)\}_{t=1}^T}~\sum_{t=1}^{T} \left[z(t) \cdot S_p(t) + (1-z(t)) \cdot S_u(t)\right]
\label{eq:cost}
\end{equation}
\end{itemize}

\end{frame}


\begin{frame}
\frametitle{Processus de décision Markovien}
\begin{center}
\begin{itemize}
\item Les états :
 
\begin{equation}
s(t) = \frac{\lfloor \mathit{SoC}(t) \cdot B \rfloor + 0.5}{B} 
\label{eq:state}
\end{equation}
\item Les actions ($a=0$, $a=1$)
\item La fonction de transition (modèle de batterie)
\item La fonction de récompense :
\begin{equation}
r(s(t),a) =
\begin{cases}
0 & \text{si $z(t)=1$ et $a=0$}\\
-C_{el}(t) \cdot \frac{E_{ch}(SoC(t))}{\eta} & \text{si $z(t)=1$ et $a=1$}\\
- C_{fuel}(t) \cdot F_{c}(SoC(t)) & \text{si $z(t)=0$}   
\end{cases}.\label{eq:reward}
\end{equation}
\end{itemize}
\end{center}
\end{frame}



\begin{frame}
\begin{center}
\frametitle{Optimisation basée sur la Programmation Dynamique}

\begin{equation}
Q(s(t),a) =
r(s(t),a) + \max\limits_{a\in\mathcal{A}} Q(s(t+1),a) 
\end{equation}

\begin{equation}
a^*(t) = \argmax\limits_{a\in\mathcal{A}}~Q(s(t),a).\label{eq:adecisions}
\end{equation}
\end{center}
\end{frame}


\begin{frame}
\frametitle{Prise de décisions avec la programmation dynamique (Exemple illustratif)}
\begin{center}
\begin{figure}
% \includegraphics[width=\linewidth]{minFct.pdf}
\includegraphics[width=0.7\linewidth]{figDinamycProgram02.pdf}
\end{figure} 
 \end{center}
\end{frame}


\begin{frame}
\frametitle{Analyse de l'intervalle de temps}
\begin{center}
\begin{figure}
% \includegraphics[width=\linewidth]{minFct.pdf}
\includegraphics[width=0.9\linewidth]{figresDeltaWithoutGasBWcolor02-Sout2.pdf}
\end{figure} 
 \end{center}
\end{frame}

\subsection{Résultats obtenus avec la programmation dynamique}

\begin{frame}
\begin{center}
\frametitle{Résultats}

% \begin{table}[t!]
% \begin{center}
% \caption[Cost and gain of \gls{dp}, \gls{ac}, and \gls{rd} models in summer and winter.]{Cost and gain regarding the \gls{gas} in relation of \gls{dp}, \gls{ac}, and \gls{rd} models in summer and winter test datasets, for the 17 \glspl{ev} evaluated. Values corresponding to the cost ($\$$) are related to the energy cost of the usage during the period, while values given as percentage ($\%$) correspond to the gain of the each strategy compared to driving only with \gls{gas}.}
% \label{t:ACmodel}
\scalebox{0.8}{
\begin{tabular}{ccccccccc}
\toprule
\multirow{3}{*}{} & \multicolumn{7}{c}{\B{Hiver}} &  \\
 & \B{Essence} & \multicolumn{2}{c}{\B{Prog. dynamique}} & \multicolumn{2}{c}{\B{Charge toujours}} & \multicolumn{2}{c}{\B{RD}} & \\
 \cmidrule(lr){3-4} \cmidrule(lr){5-6} \cmidrule(lr){7-8} 
 {} & \B{\$} & \B{\$} & \B{\%} & \B{\$} & \B{\%} & \B{\$} & \B{\%} & {}  \\
\midrule 
% 203 & 131.2 & 34.0 & 74 & \s91.0 & 31 & \s93.7 & 29 & & \s65.1 & \s7.9 & 88 & \s29.1 & 55 & \s30.1 & 54 \\
% 205 & \s93.0 & 22.1 & 76 & \s57.4 & 38 & \s57.1 & 39 & & \s45.5 & \s5.4 & 88 & \s22.2 & 51 & \s24.8 & 45 \\
% 206 & \s52.4 & \s1.5 & 97 & \s16.4 & 69 & \s14.6 & 72 & & \s34.9 & \s5.5 & 84 & \s20.5 & 41 & \s21.4 & 39 \\
% 213 & 187.4 & 15.7 & 92 & \s60.6 & 68 & \s60.4 & 68 & & \s66.5 & \s7.4 & 89 & \s27.2 & 59 & \s28.7 & 57 \\
% 215 & 230.8 & 66.8 & 71 & 165.6 & 28 & 184.1 & 20 & & \s58.8 & \s6.3 & 89 & \s24.6 & 58 & \s25.2 & 57 \\
% 216 & \s93.2 & \s3.3 & 96 & \s24.7 & 73 & \s23.9 & 74 & & \s79.2 & 19.4 & 76 & \s55.8 & 29 & \s59.1 & 25 \\
% 218 & \s81.0 & \s3.2 & 96 & \s21.4 & 74 & \s20.7 & 74 & & \s40.5 & \s4.2 & 90 & \s18.3 & 55 & \s18.8 & 53 \\
% 222 & \s66.5 & \s3.0 & 95 & \s21.0 & 68 & \s21.9 & 67 & & \s42.9 & \s4.6 & 89 & \s19.3 & 55 & \s20.8 & 52 \\
% 224 & 228.4 & 87.1 & 62 & 207.2 & \s9 & 220.8 & \s3 & & \s31.9 & \s2.8 & 91 & \s13.8 & 57 & \s13.1 & 59 \\
% 225 & \s64.4 & \s1.4 & 98 & \s17.7 & 73 & \s15.8 & 75 & & \s66.0 & \s8.4 & 87 & \s28.1 & 57 & \s29.9 & 55 \\
% 237 & 156.8 & 17.3 & 89 & \s56.5 & 64 & \s63.0 & 60 & & 108.3 & 14.2 & 87 & \s44.4 & 59 & \s47.4 & 56 \\
% 239 & \s72.2 & \s2.5 & 97 & \s23.3 & 68 & \s22.0 & 70 & & \s64.2 & \s7.4 & 88 & \s26.4 & 59 & \s28.5 & 56 \\
% 242 & 364.3 & 98.0 & 73 & 232.5 & 36 & 287.3 & 21 & & 178.0 & 49.6 & 72 & 133.8 & 25 & 150.1 & 16 \\
% 249 & 122.1 & \s5.0 & 96 & \s34.9 & 71 & \s32.9 & 73 & & \s36.1 & \s3.6 & 90 & \s15.1 & 58 & \s15.9 & 56 \\
% 262 & \s20.8 & \s0.4 & 98 & \s\s6.2 & 70 & \s\s5.8 & 72 & & \s61.8 & \s6.4 & 90 & \s25.8 & 58 & \s27.2 & 56 \\
% 263 & \s32.0 & \s2.3 & 93 & \s\s9.6 & 70 & \s\s8.3 & 74 & & \s52.8 & \s7.2 & 86 & \s23.0 & 56 & \s24.7 & 53 \\
% 265 & \s57.7 & \s1.6 & 97 & \s17.3 & 70 & \s15.7 & 73 & & \s23.8 & \s2.4 & 90 & \s10.9 & 54 & \s11.1 & 53 \\
\B{Moyenne} &  \B{\s62.1} & \B{\s9.6} & \B{87} & \B{\s31.7} & \B{52} & \B{\s33.9} & \B{50} \\
\B{Médiane} &  \B{\s58.8} & \B{\s6.4} & \B{88} & \B{\s24.6} & \B{56} & \B{\s25.2} & \B{54} \\
\bottomrule    
\end{tabular}
}

\vspace*{25px}


\scalebox{0.8}{
\begin{tabular}{ccccccccc}
\toprule
\multirow{3}{*}{} & \multicolumn{7}{c}{\B{Été}} &  \\
 & \B{Essence} & \multicolumn{2}{c}{\B{Prog. dynamique}} & \multicolumn{2}{c}{\B{Charge toujours}} & \multicolumn{2}{c}{\B{RD}} & \\
 \cmidrule(lr){3-4} \cmidrule(lr){5-6} \cmidrule(lr){7-8} 
 {} & \B{\$} & \B{\$} & \B{\%} & \B{\$} & \B{\%} & \B{\$} & \B{\%} & {}  \\
\midrule 
% 203 & 131.2 & 34.0 & 74 & \s91.0 & 31 & \s93.7 & 29 & & \s65.1 & \s7.9 & 88 & \s29.1 & 55 & \s30.1 & 54 \\
% 205 & \s93.0 & 22.1 & 76 & \s57.4 & 38 & \s57.1 & 39 & & \s45.5 & \s5.4 & 88 & \s22.2 & 51 & \s24.8 & 45 \\
% 206 & \s52.4 & \s1.5 & 97 & \s16.4 & 69 & \s14.6 & 72 & & \s34.9 & \s5.5 & 84 & \s20.5 & 41 & \s21.4 & 39 \\
% 213 & 187.4 & 15.7 & 92 & \s60.6 & 68 & \s60.4 & 68 & & \s66.5 & \s7.4 & 89 & \s27.2 & 59 & \s28.7 & 57 \\
% 215 & 230.8 & 66.8 & 71 & 165.6 & 28 & 184.1 & 20 & & \s58.8 & \s6.3 & 89 & \s24.6 & 58 & \s25.2 & 57 \\
% 216 & \s93.2 & \s3.3 & 96 & \s24.7 & 73 & \s23.9 & 74 & & \s79.2 & 19.4 & 76 & \s55.8 & 29 & \s59.1 & 25 \\
% 218 & \s81.0 & \s3.2 & 96 & \s21.4 & 74 & \s20.7 & 74 & & \s40.5 & \s4.2 & 90 & \s18.3 & 55 & \s18.8 & 53 \\
% 222 & \s66.5 & \s3.0 & 95 & \s21.0 & 68 & \s21.9 & 67 & & \s42.9 & \s4.6 & 89 & \s19.3 & 55 & \s20.8 & 52 \\
% 224 & 228.4 & 87.1 & 62 & 207.2 & \s9 & 220.8 & \s3 & & \s31.9 & \s2.8 & 91 & \s13.8 & 57 & \s13.1 & 59 \\
% 225 & \s64.4 & \s1.4 & 98 & \s17.7 & 73 & \s15.8 & 75 & & \s66.0 & \s8.4 & 87 & \s28.1 & 57 & \s29.9 & 55 \\
% 237 & 156.8 & 17.3 & 89 & \s56.5 & 64 & \s63.0 & 60 & & 108.3 & 14.2 & 87 & \s44.4 & 59 & \s47.4 & 56 \\
% 239 & \s72.2 & \s2.5 & 97 & \s23.3 & 68 & \s22.0 & 70 & & \s64.2 & \s7.4 & 88 & \s26.4 & 59 & \s28.5 & 56 \\
% 242 & 364.3 & 98.0 & 73 & 232.5 & 36 & 287.3 & 21 & & 178.0 & 49.6 & 72 & 133.8 & 25 & 150.1 & 16 \\
% 249 & 122.1 & \s5.0 & 96 & \s34.9 & 71 & \s32.9 & 73 & & \s36.1 & \s3.6 & 90 & \s15.1 & 58 & \s15.9 & 56 \\
% 262 & \s20.8 & \s0.4 & 98 & \s\s6.2 & 70 & \s\s5.8 & 72 & & \s61.8 & \s6.4 & 90 & \s25.8 & 58 & \s27.2 & 56 \\
% 263 & \s32.0 & \s2.3 & 93 & \s\s9.6 & 70 & \s\s8.3 & 74 & & \s52.8 & \s7.2 & 86 & \s23.0 & 56 & \s24.7 & 53 \\
% 265 & \s57.7 & \s1.6 & 97 & \s17.3 & 70 & \s15.7 & 73 & & \s23.8 & \s2.4 & 90 & \s10.9 & 54 & \s11.1 & 53 \\
\B{Moyenne} & \B{120.8} & \B{\s21.5} & \B{88} & \B{\s62.6} & \B{58} & \B{\s67.5} & \B{57} &   \\
\B{Médiane} & \B{\s93.0} & \B{\s3.3} & \B{96} & \B{\s24.7} & \B{68} & \B{\s23.9} & \B{70} &  \\
\bottomrule    
\end{tabular}
}

% \end{center}
% \end{table}
\end{center}
\end{frame}


% 
% \section{Système d'information}
% \subsection{Analyse univariée des données}
% \begin{frame}
% \frametitle{Analyses univariées des données} 
% \begin{overlayarea}{\linewidth}{2.8in}
% \begin{center} 
% \includegraphics<1 | handout:0>[width=0.78\linewidth]{figUnivAnlFr1.pdf} 
% \includegraphics<2 | handout:0>[width=0.78\linewidth]{figUnivAnlFr2.pdf}
% \includegraphics<3 | handout:0>[width=0.78\linewidth]{figUnivAnlFr3.pdf}
% \end{center}
% \end{overlayarea}
% \end{frame}
% 
% \subsection{Analyse multivariée des données}
% \begin{frame}
% \frametitle{Analyses multivariées des données - Corrélation et ACP} 
% 
% \begin{table}[t!]
% \begin{center}
% \centering
% \scalebox{0.65}{
% \begin{tabular}{lccccc}
% \toprule
% {} & \textbf{Température} & \textbf{Énergie consommée} & \textbf{Prix de l'essence} & \textbf{HOD} & \textbf{HOEP} \\
% \midrule
% \textbf{Température}   	& \x1.00&0.02	&\B{0.81}	&$-$0.02	&0.15	\\
% \textbf{Énergie consommée}    & 	&1.00	&0.02	&\x0.06		&0.06	\\
% \textbf{Prix de l'essence} & 	&	&1.00	&$-$0.00	&0.17	\\
% \textbf{HOD}            & 	& 	&	&\x1.00		&\B{0.71}	\\
% \textbf{HOEP}           & 	&	&	&		&1.00	\\
% \bottomrule
% \end{tabular}
% }
% \end{center}
% \end{table}
% 
% \pause
% 
% \begin{table}
% \begin{center}
% \centering
% \scalebox{0.65}{
% \begin{tabular}{lcccccc}
% \toprule
% {} & \textbf{1$^{\mbox{st}}$} & \textbf{2$^{\mbox{nd}}$} & 
% \textbf{3$^{\mbox{rd}}$} & \textbf{4$^{\mbox{th}}$} & \textbf{5$^{\mbox{th}}$} \\
% \midrule
% \textbf{Température}   &0.57&$-$0.41&$-$0.01&$-$0.06&\x0.71\\
% \textbf{Énergie consommée}    &0.08&\x0.08&\x0.99&\x0.01&\x0.00\\
% \textbf{Prix de l'essence} &0.58&$-$0.40&$-$0.02&$-$0.10&$-$0.71\\
% \textbf{HOD}                &0.35&\x0.63&$-$0.07&$-$0.69&\x0.03\\
% \textbf{HOEP}              &0.47&\x0.52&$-$0.09&\x0.71&$-$0.01\\
% \bottomrule
% \end{tabular}
% }
% \end{center}
% \end{table}
% \end{frame}
% 
% 
% \begin{frame}
% \frametitle{Définition du Système d'Information} 
% 
% 
% \begin{itemize}
% \item HOEP ($\ve{x}^1$), HOD ($\ve{x}^2$), et température extérieur ($\ve{x}^3$) with 101 décalages.
% \item énergie consommée ($\ve{x}^4$) avec 199 décalages.
% \item variables scalaires : 
% \begin{itemize}
% \item $w_1:$ jour de la semaine,
% \item $w_2:$ heure;
% \item $w_3: C_\mathit{el}(t-1)-C_\mathit{el}(t)$ différence du prix de l'électricité,
% \item $w_4: C_\mathit{el}(t)$ prix de l'électricité au temps $t$ [\$/kWh],
% \item $w_5: C_\mathit{fuel}(t)$ prix de l’essence au temps $t$[\$/l],
% \item $w_6: dis(t)$ distance parcourue au temps $t$ [km].
% \end{itemize}
% \end{itemize}
% 
% \pause
% ... et l'état de charge du véhicule ainsi que les décisions optimales obtenues avec la programmation dynamique.
% 
% \end{frame}
% 
% \begin{frame}
% \frametitle{ACP du Système d'Information} 
% \begin{figure}
% \begin{center}
% \includegraphics[width=0.99\textwidth]{figPCAvarExplainedFr.png}
% \end{center}
% \end{figure}
% \end{frame}
% 
% 
% \section{Stratification des données}
% \subsection{Méthode pour la stratification de données}
% \begin{frame}
% \frametitle{Stratification des données} 
% \begin{figure}
% \begin{center}
% \includegraphics<1>[width=0.9\linewidth]{figClusteringTrainingFr_222_summer1.pdf}
% \includegraphics<2>[width=0.9\linewidth]{figStratEstimFr_222_summer1.pdf}
% \end{center}
% \end{figure}
% \end{frame}
% % 
% \begin{frame}
% \frametitle{Stratification dans un problème de régression avec réseau de neurones} 
% \begin{figure}
% \begin{center}
% \includegraphics[width=0.7\textwidth]{figDiff_ANN_Inp_state_year_2003.pdf}
% \end{center}
% \end{figure}
% \end{frame}
% 
% \begin{frame}
% \frametitle{Stratification dans un problème de régression avec machines à vecteurs de support} 
% \begin{figure}
% \begin{center}
% \includegraphics[width=0.7\textwidth]{figDiff_SVR_Inp_state_year_2003.pdf}
% \end{center}
% \end{figure}
% \end{frame}
% % 


\section{Recharge intelligente de VE}

\subsection{Méthodologie}


 \begin{frame}{Considérations importantes}
 \begin{itemize}
 	\item Aucune hypothèse nécessaire sur les modèles utilisés
 	\begin{itemize}
 		\item Voiture hybride rechargeable / 100\% électrique
 		\item Caractéristiques de la batterie
 		\item Caractéristiques du chargeur
 		\item Incitatifs supplémentaires (e.g. vente d'électricité)
 	\end{itemize}
 	\item La solution obtenue est \emph{globalement optimale}
 \end{itemize}
 \end{frame}
 
 


\begin{frame}
\begin{center}
%\includegraphics[width=0.7\textwidth]{artMeth2CompactFr.pdf}
\huge \textbf{Apprendre les décisions optimales}
\end{center}
\end{frame}

\begin{frame}{Pipeline d'apprentissage automatique}
\begin{itemize}
	\item Mêmes données d'entrée, sans connaître le futur
	\item Vérité terrain donnée par la programmation dynamique
	\item Techniques comparées :
	\begin{itemize}
		\item Système de régles à base de seuils
		\item k-plus proches voisins
		\item Réseaux de neurones avec de données stratifiés
		 \item Réseaux de neurones profonds
 
	\end{itemize}
\end{itemize}

\end{frame}


\begin{frame}
\frametitle{Système de régles à base de seuils} 
\begin{figure}
\begin{center}
\includegraphics[width=0.9\textwidth]{figtbrFr.pdf}
\end{center}
\end{figure}
\end{frame}
% 


\begin{frame}
\frametitle{k-plus proches voisins} 
\begin{figure}
\begin{center}
\includegraphics[width=0.9\textwidth]{figknn.pdf}
\end{center}
\end{figure}
\end{frame}
% 
% 

\subsection{Analyse des résultats}
% 
% \begin{frame}
% \begin{center}
% \begin{overlayarea}{\linewidth}{2.8in}
% \includegraphics<1 | handout:0>[width=\linewidth]{resultats01.pdf}
% \includegraphics<2 | handout:0>[width=\linewidth]{resultats02.pdf}
% \includegraphics<3>[width=\linewidth]{resultats03.pdf}
% \end{overlayarea}
% \end{center}
% \end{frame}


\begin{frame}
\begin{center}
\begin{overlayarea}{\linewidth}{2.8in}
\includegraphics<1 | handout:0>[width=\linewidth]{figResAnalysis01.pdf}
\includegraphics<2 | handout:0>[width=\linewidth]{figResAnalysis02.pdf}
\includegraphics<3>[width=\linewidth]{figResAnalysis03.pdf}
\end{overlayarea}
\end{center}
\end{frame}



\begin{frame}
\frametitle{Résultats} 
\begin{figure}
\begin{center}
\begin{overlayarea}{\linewidth}{2.8in}
\includegraphics<1 | handout:0>[width=\linewidth]{figCostMLFr01.pdf}
\includegraphics<2 | handout:0>[width=\linewidth]{figCostMLFr02.pdf}
\includegraphics<3 | handout:0>[width=\linewidth]{figCostMLFr03.pdf}
\includegraphics<4 | handout:0>[width=\linewidth]{figCostMLFr04.pdf}
\includegraphics<5 | handout:0>[width=\linewidth]{figCostMLFr05.pdf}
\includegraphics<6 | handout:0>[width=\linewidth]{figCostMLFr06.pdf}
\includegraphics<7 | handout:0>[width=\linewidth]{figCostMLFr07.pdf}
\end{overlayarea}
\end{center}
\end{figure}
\end{frame}



\section{Contributions}
\begin{frame}
\frametitle{Papers in conference proceedings}
\begin{itemize}
   \item \bibentry{lopez2018b}.   
\end{itemize} 

\end{frame}

\begin{frame}
\frametitle{Journal articles} 


\begin{itemize}
  \item \bibentry{lopez2018}.
  \item \bibentry{lopez2015}.
\end{itemize}
\end{frame}


{\usebackgroundtemplate{%
  \parbox[c][\paperheight][c]{\paperwidth}{\centering\includegraphics[width=\linewidth]{quest.jpg}}} 
\begin{frame}[plain,noframenumbering]
\begin{center}
\textcolor{white}{\Large{\textbf{Merci de votre attention!}}} 
\end{center}
% \vspace{60mm}
% \textcolor{white}{\small{\textbf{\hfill Pr\'esentation faite avec \LaTeX}}}\\
% \hfill \includegraphics[width=0.2\linewidth]{fig/bf.pdf}
\end{frame}
}

\bibliographystyle{siam}
% \bibliographystyle{plainnat}
% \bibliographystyle{plain-fr}
\bibliography{references}                 % production de la bibliographie


\end{document}
}

